%------------------------------------- Anexo 1 ------------------------------------------------------
\chapter{Anexo 1}
\label{Anexo1}

Os seguintes trechos de código foram incorporados ao projeto da interface gráfica para a estação-base. Duas funcionalidades foram adicionadas ao código. A primeira funcionalidade corresponde a parte de reconstrução tridimensional e renderização da nuvem de pontos \cite{UEDA2011}. A segunda funcionalidade corresponde a parte de rastreamento do objeto desenvolvida por Kyle Hounslow \cite{Hounslow2013}. 


\lstset {language=C++}
\textbf{3DReconstruction.h}
\begin{lstlisting}[basicstyle=\tiny]
#ifndef RECONSTRUCTION_3D_H
#define RECONSTRUCTION_3D_H

/* Libraries */
#include "opencv2/opencv.hpp"

using namespace cv;
using namespace std;

/* 3D Reconstruction Classes */
template <class T>
static void projectImagefromXYZ_(Mat& image, Mat& destimage, Mat& disp, 
				 Mat& destdisp, Mat& xyz, Mat& R, Mat& t, 
				 Mat& K, Mat& dist, Mat& mask, bool isSub);

template <class T>
static void fillOcclusionInv_(Mat& src, T invalidvalue);

template <class T>
static void fillOcclusion_(Mat& src, T invalidvalue);

// 3D Reconstruction
class Reconstruction3D{
public:
    Reconstruction3D(); //Constructor
    void setViewPoint(double x,double y,double z);
    void setLookAtPoint(double x,double y,double z);
    void PointCloudInit(double baseline,bool isSub);

    /* 3D Reconstruction Functions */
    void eular2rot(double yaw,double pitch, double roll,Mat& dest);
    void lookat(Point3d from, Point3d to, Mat& destR);
    void projectImagefromXYZ(Mat &image, Mat &destimage, Mat &disp, 
			     Mat &destdisp, Mat &xyz, Mat &R, Mat &t,
			     Mat &K, Mat &dist, bool isSub);
    void fillOcclusion(Mat& src, int invalidvalue, bool isInv);


    Mat disp3Dviewer;
    Mat disp3D;
    Mat disp3D_8U;
    Mat disp3D_BGR;

    Point3d viewpoint;
    Point3d lookatpoint;


    Mat dist;
    Mat Rotation;
    Mat t;

    Mat depth;

    double step;
    bool isSub;
};


#endif // RECONSTRUCTION_3D_H
\end{lstlisting}

\textbf{trackObject.h}
\begin{lstlisting}[basicstyle=\tiny]
/*
 * trackObject.h
 *
 *  Author: Kyle Hounslow
 * 	Link: https://www.youtube.com/watch?v=bSeFrPrqZ2A
 *  Published in: March 11, 2013
 */

#ifndef SRC_TRACKOBJECT_H_
#define SRC_TRACKOBJECT_H_

/* Libraries */
#include "opencv2/opencv.hpp"

using namespace cv;
using namespace std;

//default capture width and height
const int FRAME_WIDTH = 640;
const int FRAME_HEIGHT = 480;
//max number of objects to be detected in frame
const int MAX_NUM_OBJECTS=50;
//minimum and maximum object area
const int MIN_OBJECT_AREA = 20*20;
const int MAX_OBJECT_AREA = FRAME_HEIGHT*FRAME_WIDTH/1.5;

string intToString(int number);
void drawObject(int x, int y,Mat &frame);
void trackFilteredObject(int &x, int &y, Mat threshold, Mat &cameraFeed);

#endif /* SRC_TRACKOBJECT_H_ */
\end{lstlisting}

\textbf{3DReconstruction.cpp}
\begin{lstlisting}[basicstyle=\tiny]
/*
 * 3DReconstruction.cpp
 *
 *  Created on: Oct 25, 2015
 *      Author: nicolasrosa
 */

#include "3DReconstruction.h"

/* Constructor */
Reconstruction3D::Reconstruction3D(){}

void Reconstruction3D::setViewPoint(double x,double y,double z){
    this->viewpoint.x = x;
    this->viewpoint.y = y;
    this->viewpoint.z = z;
}

void Reconstruction3D::setLookAtPoint(double x,double y,double z){
    this->lookatpoint.x = x;
    this->lookatpoint.y = y;
    this->lookatpoint.z = z;
}

void Reconstruction3D::PointCloudInit(double baseline,bool isSub){
    this->dist=Mat::zeros(5,1,CV_64F);
    this->Rotation=Mat::eye(3,3,CV_64F);
    this->t=Mat::zeros(3,1,CV_64F);

    this->isSub = isSub;
    this->step = baseline/10;
}

void Reconstruction3D::eular2rot(double yaw,double pitch, double roll,Mat& dest){
    double theta = yaw/180.0*CV_PI;
    double pusai = pitch/180.0*CV_PI;
    double phi = roll/180.0*CV_PI;

    double datax[3][3] = {{1.0,0.0,0.0},{0.0,cos(theta),-sin(theta)},{0.0,sin(theta),cos(theta)}};
    double datay[3][3] = {{cos(pusai),0.0,sin(pusai)},{0.0,1.0,0.0},{-sin(pusai),0.0,cos(pusai)}};
    double dataz[3][3] = {{cos(phi),-sin(phi),0.0},{sin(phi),cos(phi),0.0},{0.0,0.0,1.0}};
    Mat Rx(3,3,CV_64F,datax);
    Mat Ry(3,3,CV_64F,datay);
    Mat Rz(3,3,CV_64F,dataz);
    Mat rr=Rz*Rx*Ry;
    rr.copyTo(dest);
}

void Reconstruction3D::lookat(Point3d from, Point3d to, Mat& destR){
    double x=(to.x-from.x);
    double y=(to.y-from.y);
    double z=(to.z-from.z);

    double pitch =asin(x/sqrt(x*x+z*z))/CV_PI*180.0;
    double yaw =asin(-y/sqrt(y*y+z*z))/CV_PI*180.0;

    eular2rot(yaw, pitch, 0,destR);
}

void Reconstruction3D::projectImagefromXYZ(Mat &image, Mat &destimage, Mat &disp, 
					   Mat &destdisp, Mat &xyz, Mat &R, Mat &t, 
					   Mat &K, Mat &dist, bool isSub){
    Mat mask;
    if(mask.empty())mask=Mat::zeros(image.size(),CV_8U);
    if(disp.type()==CV_8U){
        projectImagefromXYZ_<unsigned char>(image,destimage, disp, destdisp, xyz, R, t, K, dist, mask,isSub);
    }
    else if(disp.type()==CV_16S){
        projectImagefromXYZ_<short>(image,destimage, disp, destdisp, xyz, R, t, K, dist, mask,isSub);
    }
    else if(disp.type()==CV_16U){
        projectImagefromXYZ_<unsigned short>(image,destimage, disp, destdisp, xyz, R, t, K, dist, mask,isSub);
    }
    else if(disp.type()==CV_32F){
        projectImagefromXYZ_<float>(image,destimage, disp, destdisp, xyz, R, t, K, dist, mask,isSub);
    }
    else if(disp.type()==CV_64F){
        projectImagefromXYZ_<double>(image,destimage, disp, destdisp, xyz, R, t, K, dist, mask,isSub);
    }
}

template <class T>
static void fillOcclusionInv_(Mat& src, T invalidvalue){
    int bb=1;
    const int MAX_LENGTH=src.cols*0.8;
    //#pragma omp parallel for
    for(int j=bb;j<src.rows-bb;j++){
        T* s = src.ptr<T>(j);
        //const T st = s[0];
        //const T ed = s[src.cols-1];
        s[0]=0;
        s[src.cols-1]=0;
        for(int i=0;i<src.cols;i++){
            if(s[i]==invalidvalue){
                int t=i;
                do{
                    t++;
                    if(t>src.cols-1)break;
                }while(s[t]==invalidvalue);

                const T dd = max(s[i-1],s[t]);
                if(t-i>MAX_LENGTH){
                    for(int n=0;n<src.cols;n++){
                        s[n]=invalidvalue;
                    }
                }
                else{
                    for(;i<t;i++){
                        s[i]=dd;
                    }
                }
            }
        }
    }
}

template <class T>
static void projectImagefromXYZ_(Mat& image, Mat& destimage, Mat& disp, 
				 Mat& destdisp, Mat& xyz, Mat& R, Mat& t, 
				 Mat& K, Mat& dist, Mat& mask, bool isSub){
    if(destimage.empty())destimage=Mat::zeros(Size(image.size()),image.type());
    if(destdisp.empty())destdisp=Mat::zeros(Size(image.size()),disp.type());

    vector<Point2f> pt;
    if(dist.empty()) dist = Mat::zeros(Size(5,1),CV_32F);
    cv::projectPoints(xyz,R,t,K,dist,pt);
    destimage.setTo(0);
    destdisp.setTo(0);

    //#pragma omp parallel for
    for(int j=1;j<image.rows-1;j++){
        int count=j*image.cols;
        uchar* img=image.ptr<uchar>(j);
        uchar* m=mask.ptr<uchar>(j);
        for(int i=0;i<image.cols;i++,count++){
            int x=(int)(pt[count].x+0.5);
            int y=(int)(pt[count].y+0.5);
            if(m[i]==255)continue;
            if(pt[count].x>=1 && pt[count].x<image.cols-1 && pt[count].y>=1 && pt[count].y<image.rows-1){
                short v=destdisp.at<T>(y,x);
                if(v<disp.at<T>(j,i)){
                    destimage.at<uchar>(y,3*x+0)=img[3*i+0];
                    destimage.at<uchar>(y,3*x+1)=img[3*i+1];
                    destimage.at<uchar>(y,3*x+2)=img[3*i+2];
                    destdisp.at<T>(y,x)=disp.at<T>(j,i);

                    if(isSub){
                        if((int)pt[count+image.cols].y-y>1 && (int)pt[count+1].x-x>1){
                            destimage.at<uchar>(y,3*x+3)=img[3*i+0];
                            destimage.at<uchar>(y,3*x+4)=img[3*i+1];
                            destimage.at<uchar>(y,3*x+5)=img[3*i+2];

                            destimage.at<uchar>(y+1,3*x+0)=img[3*i+0];
                            destimage.at<uchar>(y+1,3*x+1)=img[3*i+1];
                            destimage.at<uchar>(y+1,3*x+2)=img[3*i+2];

                            destimage.at<uchar>(y+1,3*x+3)=img[3*i+0];
                            destimage.at<uchar>(y+1,3*x+4)=img[3*i+1];
                            destimage.at<uchar>(y+1,3*x+5)=img[3*i+2];

                            destdisp.at<T>(y,x+1)=disp.at<T>(j,i);
                            destdisp.at<T>(y+1,x)=disp.at<T>(j,i);
                            destdisp.at<T>(y+1,x+1)=disp.at<T>(j,i);
                        }
                        else if((int)pt[count-image.cols].y-y<-1 && (int)pt[count-1].x-x<-1){
                            destimage.at<uchar>(y,3*x-3)=img[3*i+0];
                            destimage.at<uchar>(y,3*x-2)=img[3*i+1];
                            destimage.at<uchar>(y,3*x-1)=img[3*i+2];

                            destimage.at<uchar>(y-1,3*x+0)=img[3*i+0];
                            destimage.at<uchar>(y-1,3*x+1)=img[3*i+1];
                            destimage.at<uchar>(y-1,3*x+2)=img[3*i+2];

                            destimage.at<uchar>(y-1,3*x-3)=img[3*i+0];
                            destimage.at<uchar>(y-1,3*x-2)=img[3*i+1];
                            destimage.at<uchar>(y-1,3*x-1)=img[3*i+2];

                            destdisp.at<T>(y,x-1)=disp.at<T>(j,i);
                            destdisp.at<T>(y-1,x)=disp.at<T>(j,i);
                            destdisp.at<T>(y-1,x-1)=disp.at<T>(j,i);
                        }
                        else if((int)pt[count+1].x-x>1){
                            destimage.at<uchar>(y,3*x+3)=img[3*i+0];
                            destimage.at<uchar>(y,3*x+4)=img[3*i+1];
                            destimage.at<uchar>(y,3*x+5)=img[3*i+2];

                            destdisp.at<T>(y,x+1)=disp.at<T>(j,i);
                        }
                        else if((int)pt[count-1].x-x<-1){
                            destimage.at<uchar>(y,3*x-3)=img[3*i+0];
                            destimage.at<uchar>(y,3*x-2)=img[3*i+1];
                            destimage.at<uchar>(y,3*x-1)=img[3*i+2];

                            destdisp.at<T>(y,x-1)=disp.at<T>(j,i);
                        }
                        else if((int)pt[count+image.cols].y-y>1){
                            destimage.at<uchar>(y+1,3*x+0)=img[3*i+0];
                            destimage.at<uchar>(y+1,3*x+1)=img[3*i+1];
                            destimage.at<uchar>(y+1,3*x+2)=img[3*i+2];

                            destdisp.at<T>(y+1,x)=disp.at<T>(j,i);
                        }
                        else if((int)pt[count-image.cols].y-y<-1){
                            destimage.at<uchar>(y-1,3*x+0)=img[3*i+0];
                            destimage.at<uchar>(y-1,3*x+1)=img[3*i+1];
                            destimage.at<uchar>(y-1,3*x+2)=img[3*i+2];

                            destdisp.at<T>(y-1,x)=disp.at<T>(j,i);
                        }
                    }
                }
            }
        }
    }

    if(isSub)
    {
        Mat image2;
        Mat disp2;
        destimage.copyTo(image2);
        destdisp.copyTo(disp2);
        const int BS=1;
        //#pragma omp parallel for
        for(int j=BS;j<image.rows-BS;j++){
            uchar* img=destimage.ptr<uchar>(j);
            T* m = disp2.ptr<T>(j);
            T* dp = destdisp.ptr<T>(j);
            for(int i=BS;i<image.cols-BS;i++){
                if(m[i]==0){
                    int count=0;
                    int d=0;
                    int r=0;
                    int g=0;
                    int b=0;
                    for(int l=-BS;l<=BS;l++){
                        T* dp2 = disp2.ptr<T>(j+l);
                        uchar* imageR = image2.ptr<uchar>(j+l);
                        for(int k=-BS;k<=BS;k++){
                            if(dp2[i+k]!=0){
                                count++;
                                d+=dp2[i+k];
                                r+=imageR[3*(i+k)+0];
                                g+=imageR[3*(i+k)+1];
                                b+=imageR[3*(i+k)+2];
                            }
                        }
                    }
                    if(count!=0){
                        double div = 1.0/count;
                        dp[i]=d*div;
                        img[3*i+0]=r*div;
                        img[3*i+1]=g*div;
                        img[3*i+2]=b*div;
                    }
                }
            }
        }
    }
}

void fillOcclusion(Mat& src, int invalidvalue, bool isInv){
    if(isInv){
        if(src.type()==CV_8U){
            fillOcclusionInv_<uchar>(src, (uchar)invalidvalue);
        }
        else if(src.type()==CV_16S){
            fillOcclusionInv_<short>(src, (short)invalidvalue);
        }
        else if(src.type()==CV_16U){
            fillOcclusionInv_<unsigned short>(src, (unsigned short)invalidvalue);
        }
        else if(src.type()==CV_32F){
            fillOcclusionInv_<float>(src, (float)invalidvalue);
        }
    }
    else{
        if(src.type()==CV_8U){
            fillOcclusion_<uchar>(src, (uchar)invalidvalue);
        }
        else if(src.type()==CV_16S){
            fillOcclusion_<short>(src, (short)invalidvalue);
        }
        else if(src.type()==CV_16U){
            fillOcclusion_<unsigned short>(src, (unsigned short)invalidvalue);
        }
        else if(src.type()==CV_32F){
            fillOcclusion_<float>(src, (float)invalidvalue);
        }
    }
}

template <class T>
static void fillOcclusion_(Mat& src, T invalidvalue){
    int bb=1;
    const int MAX_LENGTH=src.cols*0.5;
    //#pragma omp parallel for
    for(int j=bb;j<src.rows-bb;j++){
        T* s = src.ptr<T>(j);
        //const T st = s[0];
        //const T ed = s[src.cols-1];
        s[0]=255;
        s[src.cols-1]=255;
        for(int i=0;i<src.cols;i++){
            if(s[i]<=invalidvalue){
                int t=i;
                do{
                    t++;
                    if(t>src.cols-1)break;
                }while(s[t]<=invalidvalue);

                const T dd = min(s[i-1],s[t]);
                if(t-i>MAX_LENGTH){
                    for(int n=0;n<src.cols;n++){
                        s[n]=invalidvalue;
                    }
                }
                else{
                    for(;i<t;i++){
                        s[i]=dd;
                    }
                }
            }
        }
    }
}
\end{lstlisting}

\textbf{trackObject.cpp}
\begin{lstlisting}[basicstyle=\tiny]
/*
 * trackObject.cpp
 *
 *  Created on: Oct 19, 2015
 *      Author: nicolasrosa
 */

#include "trackObject.h"

void trackFilteredObject(int &x, int &y, Mat threshold, Mat &cameraFeed){

	Mat temp;
	threshold.copyTo(temp);
	//these two vectors needed for output of findContours
	std::vector< std::vector<Point> > contours;
	std::vector<Vec4i> hierarchy;
	//find contours of filtered image using openCV findContours function
	findContours(temp,contours,hierarchy,CV_RETR_CCOMP,CV_CHAIN_APPROX_SIMPLE );
	//use moments method to find our filtered object
	double refArea = 0;
	bool objectFound = false;
	if (hierarchy.size() > 0) {
		int numObjects = hierarchy.size();
        //if number of objects greater than MAX_NUM_OBJECTS we have a noisy filter
        if(numObjects<MAX_NUM_OBJECTS){
			for (int index = 0; index >= 0; index = hierarchy[index][0]) {

				Moments moment = moments((cv::Mat)contours[index]);
				double area = moment.m00;

				//if the area is less than 20 px by 20px then it is probably just noise
				//if the area is the same as the 3/2 of the image size, probably just a bad filter
				//we only want the object with the largest area so we safe a reference area each
				//iteration and compare it to the area in the next iteration.
                if(area>MIN_OBJECT_AREA && area<MAX_OBJECT_AREA && area>refArea){
					x = moment.m10/area;
					y = moment.m01/area;
					objectFound = true;
					refArea = area;
				}else objectFound = false;


			}
			//let user know you found an object
			if(objectFound ==true){
				putText(cameraFeed,"Tracking Object",Point(0,50),2,1,Scalar(0,255,0),2);
				//draw object location on screen
				drawObject(x,y,cameraFeed);}

		}else putText(cameraFeed,"TOO MUCH NOISE! ADJUST FILTER",Point(0,50),1,2,Scalar(0,0,255),2);
	}
}

void drawObject(int x, int y,Mat &frame){

	//use some of the openCV drawing functions to draw crosshairs
	//on your tracked image!

    //UPDATE:JUNE 18TH, 2013
    //added 'if' and 'else' statements to prevent
    //memory errors from writing off the screen (ie. (-25,-25) is not within the window!)

	circle(frame,Point(x,y),20,Scalar(0,255,0),2);
    if(y-25>0)
    line(frame,Point(x,y),Point(x,y-25),Scalar(0,255,0),2);
    else line(frame,Point(x,y),Point(x,0),Scalar(0,255,0),2);
    if(y+25<FRAME_HEIGHT)
    line(frame,Point(x,y),Point(x,y+25),Scalar(0,255,0),2);
    else line(frame,Point(x,y),Point(x,FRAME_HEIGHT),Scalar(0,255,0),2);
    if(x-25>0)
    line(frame,Point(x,y),Point(x-25,y),Scalar(0,255,0),2);
    else line(frame,Point(x,y),Point(0,y),Scalar(0,255,0),2);
    if(x+25<FRAME_WIDTH)
    line(frame,Point(x,y),Point(x+25,y),Scalar(0,255,0),2);
    else line(frame,Point(x,y),Point(FRAME_WIDTH,y),Scalar(0,255,0),2);

	putText(frame,intToString(x)+","+intToString(y),Point(x,y+30),1,1,Scalar(0,255,0),2);

}

string intToString(int number){
    stringstream ss;
	ss << number;
	return ss.str();
}
\end{lstlisting}