%-----------------------------------------------------------------------------------------------------------------------------------------------------------------------------------------------
% Conclusões: "fecha" com os objetivos? (respondem aos objetivos?) - aqui é que "se vende o peixe" - elas é que valorizam (ou não) o trabalho realizado.
% Normalmente é uma parte do trabalho "um pouco desprezada", pois o autor já está "cansado....". Mas é aqui que realmente se mede se o trabalho tem ou não valor.
% Contém o item Trabalhos futuros, que é uma orientação sobre as possibilidades de continuação do desenvolvimento do trabalho.
\chapter{Conclusão e Trabalhos Futuros}
\label{Conclusao}

\section{Conclusão}
O trabalho cumpriu todos os objetivos apresentados na seção \ref{objetivos}. 

No que diz respeito à visão estéreo, foram realizados o estudo e a aplicação das técnicas de visão computacional para o problema proposto, mais precisamente estudou-se os processos de calibração, retificação, filtragem e os métodos estéreo para a obtenção de um mapa de disparidades denso. 

No que concerne o monitoramento de um veículo autônomo, foi desenvolvida uma plataforma de apoio que facilita o processo de calibração dos métodos estéreo para as plataformas embarcadas, visto que são destinadas à uma aplicação que não necessita de interface gráfica. Além disso, ela possibilita o monitoramento via estação-base, caso deseja-se que o processamento das imagens não seja realizado \textit{on-board}. Os métodos foram implementados e testados em cenários capturados pela câmera fixada no quadricóptero. Entretanto, a extensão desse trabalho não se resume a veículos aéreos, as plataformas utilizadas podem muito bem serem utilizadas em veículos terrestres e até mesmo em veículos aquáticos, evidentemente com a devida vedação dos equipamentos. Outro importante objetivo era realização comparação de desempenho dos métodos nas plataformas disponíveis, BBB e Jetson TK1. No caso da BBB, mesmo a rotina desenvolvida mostrando-se totalmente funcional, isto é, os algoritmos consegue detectar e segmentar relativamente bem os obstáculos dos cenários propostos, a taxa de processamento tornou-se uma preocupação. O desempenho foi extremamente baixo, aproximadamente um quadro por segundo, o que claramente inviabiliza sua utilização em uma aplicação real. A alteração de plataforma mostrava-se ser uma solução para o aumento da taxa de atualização, visto que a Jetson TK1 é uma plataforma mais recente e mais poderosa. O desempenho desenvolvido pela Jetson TK1 era de aproximadamente seis FPS (Método BM), performance ainda baixa para um sistema que deve apresentar resposta em tempo real.

Diante deste impasse, duas providências foram cruciais para a continuidade do trabalho. A primeira foi a otimização das rotinas utilizadas, sendo essa uma tentativa para a melhora do desempenho geral dos métodos. A segunda foi a execução de uma ampla revisão bibliográfica, incluindo principalmente trabalhos que apresentassem comparativos de desempenho entre plataformas e que tivessem realizados algum tipo de aceleração por \textit{hardware}, sejam eles envolvendo paralelização de processos, implementação em FPGA ou utilizando a plataforma de computação paralela CUDA. Ao fim, a última alternativa foi tomada visto que a plataforma Jetson TK1 apresenta suporte à tecnologia CUDA.

Com relação à utilização de uma plataforma embarcada em quadricóptero, a plataforma Jetson TK1 juntamente com o método BMGPU foi a melhor alternativa para a implementação em um sistema real, mesmo apresentando uma taxa de performance reduzida, era a melhor solução existente durante o período de desenvolvimento do trabalho. Vale ressaltar que as rotinas implementadas utilizaram os métodos estéreo presentes no OpenCV, isto é, não se desenvolveu rotinas em programação CUDA, diretivas para a manipulação de dados em GPU. Deste modo, existe a expectativa de que essas bibliotecas sejam otimizadas em novas versões do OpenCV ou que a NVIDIA atualize a versão do CUDA \textit{Toolkit}, reduzindo assim o tempo de execução.

Quanto à qualidade do mapa de disparidades, a câmera utilizada, FinePix 3D W3, mesmo sendo uma câmera comercial que não necessariamente precisa se preocupar com o perfeito alinhamento do \textit{Stereo Rig} e qualidade das lentes apresentou resultados satisfatórios. Com o devido processo de calibração foi capaz de capturar os vídeos utilizados para o desenvolvimento desse trabalho, além de cumprir todos os requisitos do projeto (Leve, \textit{Baseline} rígida, Taxa de quadros elevada).

Em suma, o projeto atendeu seus principais objetivos e proporcionou o estudo de diferentes conceitos essencialmente relacionados com visão computacional, sistemas embarcados e aceleração em GPU. Obteve-se resultados expressivos com relação à qualidade dos mapas de disparidades gerados pelas implementações dos métodos estéreo, tanto em CPU quanto em GPU. Entretanto, ainda se apresenta problemas referentes à performance do sistema. Por hora, eles podem ser resolvidos com a substituição dos equipamentos por melhores câmeras ou plataformas embarcadas mais poderosas. Neste presente momento, o trabalho encontra-se em situação apta para a incorporação com sistemas robóticos mais conhecidos como \textit{Player/Stage} \cite{Gerkey2010}, \textit{Gazebo} \cite{Gazebo}, \textit{ROS} \cite{ROS}, dentre outros. Os resultados obtidos permitem que uma nova fase de pesquisa se inicie e também possibilita o estudo de novos conceitos como SLAM, algoritmos para desvio de obstáculos, planejamento de rotas.

\section{Trabalhos Futuros}

Concluiu-se o trabalho desenvolvido apresentou bons resultados em diversos aspectos, porém esses resultados podem ser melhorados através da simples troca ou atualização das câmeras e plataformas utilizadas. Além disso, ao fim do desenvolvimento do projeto, indagou-se quais modificações deveriam ser realizadas para a real implementação do sistema autônomo. Consequentemente, a maior parte das sugestões de atividades futuras descritas nessa seção estão diretamente relacionadas com as possíveis melhorias e implementação de novos conceitos.

No que se refere à atualização de plataformas, a NVIDIA lançou recentemente uma nova plataforma - Jetson TX1 que apresenta um aumento expressivo de núcleos CUDA, aproximadamente 30\% em relação à versão anterior \cite{JetsonTX1}. A implementação do trabalho desenvolvido nesta nova plataforma é bastante válida e interessante, visto que existe a possibilidade da melhora na performance do sistema. O estudo de novas alternativas de aceleração em \textit{hardware} usando, por exemplo, FPGAs que possibilitam o desenvolvimento de arquiteturas totalmente especializadas para esta aplicação \cite{Barry2015}. Outra alternativa ainda mais recente, seria a utilização da tecnologia presente nos processadores gráficos Mali™ da ARM \textregistered. Eles também permitem a aceleração em GPU, com um levado grau de especialização, através da utilização de programação paralela OpenCL \cite{StereoARM}. Ambas arquiteturas apoiam-se na utilização de \textit{pipelines}, as quais proporcionam que mais de um processo sejam executados comitantemente, o que reduz drasticamente o tempo para a execução de cada quadro. Embora a Jetson TK1 suporte à tecnologia CUDA, ela ainda é uma plataforma de desenvolvimento e de propósito geral, por conta disso, acredita-se que o desempenho das alternativas apresentadas sejam melhores.

No que diz respeito à atualização das câmeras, existem atualmente no mercado algumas câmeras que também cumprem todos os requisitos do projeto e poderia muito bem serem incorporadas. As câmeras estéro mais indicadas para esse tipo de aplicação são a \textit{Bumblebee\textregistered2} da \textit{Point Grey Research} \cite{bumblebee2}, \textit{ZED™} da \textit{StereoLabs} \cite{StereoLabsZED} e as câmeras \textit{DUO 3D™} da \textit{Code Laboratories} \cite{CodeLaboratoriesDUO}. Cada qual possui sua especialidade: Alta taxa de quadros, Alta Resolução, Tamanho Pequeno, respectivamente. Atualmente, a StereoLab e a NVIDIA têm colaborado para o desenvolvimento de sistemas que utilizem a câmera ZED™ e as plataformas da família Jetson, o que realmente incentiva a comunidade de desenvolvedores optarem por essa solução \cite{NVIDIAStereoLabsPartenership}. Isso é um indicativo que o caminho para o desenvolvimento de plataformas embarcadas utilizando visão estéreo será através da combinação de equipamentos dessas empresas.

